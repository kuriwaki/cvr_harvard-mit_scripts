% Options for packages loaded elsewhere
\PassOptionsToPackage{unicode}{hyperref}
\PassOptionsToPackage{hyphens}{url}
\PassOptionsToPackage{dvipsnames,svgnames,x11names}{xcolor}
%
\documentclass[fleqn,10pt]{wlscirep}

\usepackage{amsmath,amssymb}
\usepackage{iftex}
\ifPDFTeX
  \usepackage[T1]{fontenc}
  \usepackage[utf8]{inputenc}
  \usepackage{textcomp} % provide euro and other symbols
\else % if luatex or xetex
  \usepackage{unicode-math}
  \defaultfontfeatures{Scale=MatchLowercase}
  \defaultfontfeatures[\rmfamily]{Ligatures=TeX,Scale=1}
\fi
\usepackage{lmodern}
\ifPDFTeX\else  
    % xetex/luatex font selection
\fi
% Use upquote if available, for straight quotes in verbatim environments
\IfFileExists{upquote.sty}{\usepackage{upquote}}{}
\IfFileExists{microtype.sty}{% use microtype if available
  \usepackage[]{microtype}
  \UseMicrotypeSet[protrusion]{basicmath} % disable protrusion for tt fonts
}{}
\makeatletter
\@ifundefined{KOMAClassName}{% if non-KOMA class
  \IfFileExists{parskip.sty}{%
    \usepackage{parskip}
  }{% else
    \setlength{\parindent}{0pt}
    \setlength{\parskip}{6pt plus 2pt minus 1pt}}
}{% if KOMA class
  \KOMAoptions{parskip=half}}
\makeatother
\usepackage{xcolor}
\setlength{\emergencystretch}{3em} % prevent overfull lines
\setcounter{secnumdepth}{-\maxdimen} % remove section numbering
% Make \paragraph and \subparagraph free-standing
\ifx\paragraph\undefined\else
  \let\oldparagraph\paragraph
  \renewcommand{\paragraph}[1]{\oldparagraph{#1}\mbox{}}
\fi
\ifx\subparagraph\undefined\else
  \let\oldsubparagraph\subparagraph
  \renewcommand{\subparagraph}[1]{\oldsubparagraph{#1}\mbox{}}
\fi

\usepackage{color}
\usepackage{fancyvrb}
\newcommand{\VerbBar}{|}
\newcommand{\VERB}{\Verb[commandchars=\\\{\}]}
\DefineVerbatimEnvironment{Highlighting}{Verbatim}{commandchars=\\\{\}}
% Add ',fontsize=\small' for more characters per line
\usepackage{framed}
\definecolor{shadecolor}{RGB}{241,243,245}
\newenvironment{Shaded}{\begin{snugshade}}{\end{snugshade}}
\newcommand{\AlertTok}[1]{\textcolor[rgb]{0.68,0.00,0.00}{#1}}
\newcommand{\AnnotationTok}[1]{\textcolor[rgb]{0.37,0.37,0.37}{#1}}
\newcommand{\AttributeTok}[1]{\textcolor[rgb]{0.40,0.45,0.13}{#1}}
\newcommand{\BaseNTok}[1]{\textcolor[rgb]{0.68,0.00,0.00}{#1}}
\newcommand{\BuiltInTok}[1]{\textcolor[rgb]{0.00,0.23,0.31}{#1}}
\newcommand{\CharTok}[1]{\textcolor[rgb]{0.13,0.47,0.30}{#1}}
\newcommand{\CommentTok}[1]{\textcolor[rgb]{0.37,0.37,0.37}{#1}}
\newcommand{\CommentVarTok}[1]{\textcolor[rgb]{0.37,0.37,0.37}{\textit{#1}}}
\newcommand{\ConstantTok}[1]{\textcolor[rgb]{0.56,0.35,0.01}{#1}}
\newcommand{\ControlFlowTok}[1]{\textcolor[rgb]{0.00,0.23,0.31}{#1}}
\newcommand{\DataTypeTok}[1]{\textcolor[rgb]{0.68,0.00,0.00}{#1}}
\newcommand{\DecValTok}[1]{\textcolor[rgb]{0.68,0.00,0.00}{#1}}
\newcommand{\DocumentationTok}[1]{\textcolor[rgb]{0.37,0.37,0.37}{\textit{#1}}}
\newcommand{\ErrorTok}[1]{\textcolor[rgb]{0.68,0.00,0.00}{#1}}
\newcommand{\ExtensionTok}[1]{\textcolor[rgb]{0.00,0.23,0.31}{#1}}
\newcommand{\FloatTok}[1]{\textcolor[rgb]{0.68,0.00,0.00}{#1}}
\newcommand{\FunctionTok}[1]{\textcolor[rgb]{0.28,0.35,0.67}{#1}}
\newcommand{\ImportTok}[1]{\textcolor[rgb]{0.00,0.46,0.62}{#1}}
\newcommand{\InformationTok}[1]{\textcolor[rgb]{0.37,0.37,0.37}{#1}}
\newcommand{\KeywordTok}[1]{\textcolor[rgb]{0.00,0.23,0.31}{#1}}
\newcommand{\NormalTok}[1]{\textcolor[rgb]{0.00,0.23,0.31}{#1}}
\newcommand{\OperatorTok}[1]{\textcolor[rgb]{0.37,0.37,0.37}{#1}}
\newcommand{\OtherTok}[1]{\textcolor[rgb]{0.00,0.23,0.31}{#1}}
\newcommand{\PreprocessorTok}[1]{\textcolor[rgb]{0.68,0.00,0.00}{#1}}
\newcommand{\RegionMarkerTok}[1]{\textcolor[rgb]{0.00,0.23,0.31}{#1}}
\newcommand{\SpecialCharTok}[1]{\textcolor[rgb]{0.37,0.37,0.37}{#1}}
\newcommand{\SpecialStringTok}[1]{\textcolor[rgb]{0.13,0.47,0.30}{#1}}
\newcommand{\StringTok}[1]{\textcolor[rgb]{0.13,0.47,0.30}{#1}}
\newcommand{\VariableTok}[1]{\textcolor[rgb]{0.07,0.07,0.07}{#1}}
\newcommand{\VerbatimStringTok}[1]{\textcolor[rgb]{0.13,0.47,0.30}{#1}}
\newcommand{\WarningTok}[1]{\textcolor[rgb]{0.37,0.37,0.37}{\textit{#1}}}

\providecommand{\tightlist}{%
  \setlength{\itemsep}{0pt}\setlength{\parskip}{0pt}}\usepackage{longtable,booktabs,array}
\usepackage{calc} % for calculating minipage widths
% Correct order of tables after \paragraph or \subparagraph
\usepackage{etoolbox}
\makeatletter
\patchcmd\longtable{\par}{\if@noskipsec\mbox{}\fi\par}{}{}
\makeatother
% Allow footnotes in longtable head/foot
\IfFileExists{footnotehyper.sty}{\usepackage{footnotehyper}}{\usepackage{footnote}}
\makesavenoteenv{longtable}
\usepackage{graphicx}
\makeatletter
\def\maxwidth{\ifdim\Gin@nat@width>\linewidth\linewidth\else\Gin@nat@width\fi}
\def\maxheight{\ifdim\Gin@nat@height>\textheight\textheight\else\Gin@nat@height\fi}
\makeatother
% Scale images if necessary, so that they will not overflow the page
% margins by default, and it is still possible to overwrite the defaults
% using explicit options in \includegraphics[width, height, ...]{}
\setkeys{Gin}{width=\maxwidth,height=\maxheight,keepaspectratio}
% Set default figure placement to htbp
\makeatletter
\def\fps@figure{htbp}
\makeatother


\usepackage[utf8]{inputenc}
\usepackage[T1]{fontenc}
\usepackage{hyperref}       % hyperlinks
\usepackage{url}            % simple URL typesetting
\usepackage{booktabs}       % professional-quality tables
\usepackage{lineno}
\linenumbers
\usepackage{booktabs}
\usepackage{longtable}
\usepackage{array}
\usepackage{multirow}
\usepackage{wrapfig}
\usepackage{float}
\usepackage{colortbl}
\usepackage{pdflscape}
\usepackage{tabu}
\usepackage{threeparttable}
\usepackage{threeparttablex}
\usepackage[normalem]{ulem}
\usepackage{makecell}
\usepackage{xcolor}
\makeatletter
\@ifpackageloaded{caption}{}{\usepackage{caption}}
\AtBeginDocument{%
\ifdefined\contentsname
  \renewcommand*\contentsname{Table of contents}
\else
  \newcommand\contentsname{Table of contents}
\fi
\ifdefined\listfigurename
  \renewcommand*\listfigurename{List of Figures}
\else
  \newcommand\listfigurename{List of Figures}
\fi
\ifdefined\listtablename
  \renewcommand*\listtablename{List of Tables}
\else
  \newcommand\listtablename{List of Tables}
\fi
\ifdefined\figurename
  \renewcommand*\figurename{Figure}
\else
  \newcommand\figurename{Figure}
\fi
\ifdefined\tablename
  \renewcommand*\tablename{Table}
\else
  \newcommand\tablename{Table}
\fi
}
\@ifpackageloaded{float}{}{\usepackage{float}}
\floatstyle{ruled}
\@ifundefined{c@chapter}{\newfloat{codelisting}{h}{lop}}{\newfloat{codelisting}{h}{lop}[chapter]}
\floatname{codelisting}{Listing}
\newcommand*\listoflistings{\listof{codelisting}{List of Listings}}
\makeatother
\makeatletter
\makeatother
\makeatletter
\@ifpackageloaded{caption}{}{\usepackage{caption}}
\@ifpackageloaded{subcaption}{}{\usepackage{subcaption}}
\makeatother
\ifLuaTeX
  \usepackage{selnolig}  % disable illegal ligatures
\fi
\usepackage{bookmark}

\IfFileExists{xurl.sty}{\usepackage{xurl}}{} % add URL line breaks if available
\urlstyle{same} % disable monospaced font for URLs
\hypersetup{
  pdftitle={SciData Code example},
  pdfauthor={John Doe},
  colorlinks=true,
  linkcolor={blue},
  filecolor={Maroon},
  citecolor={Blue},
  urlcolor={Blue},
  pdfcreator={LaTeX via pandoc}}

\title{SciData Code example}

  \author[1,2,*]{John Doe}

\affil[1]{Department of Statistics}
\affil[2]{Department of Government}
\affil[*]{\texttt{correspondingauthor@email.com}}


\begin{abstract}
abstract
\end{abstract}


\begin{document}
\flushbottom


\maketitle

\thispagestyle{empty}
\paragraph{Reading in the Data}

We recommend using the \texttt{open\_dataset()} function in
\texttt{arrow} to connect to the parquet files. Parquet files are a file
storage format optimized for querying subsets of large datasets. It is
both partitioned by grouping variables, and columnar (so that users do
not need to read in an entire row to extract a value from one column).\\
Our dataset is prohibitively large to read and write in a plain-text
format (20 Gb), but is compact (800 Mb) and easy to read from in
parquet.\footnote{For more information on how to read and write parquet
  files in R, see \url{https://r4ds.hadley.nz/arrow}. Parquet is also
  designed for usage in Python
  (\url{https://arrow.apache.org/docs/python/parquet.html}) and several
  other programming languages.}

\begin{Shaded}
\begin{Highlighting}[]
\FunctionTok{library}\NormalTok{(tidyverse)}
\FunctionTok{library}\NormalTok{(arrow)}

\NormalTok{ds }\OtherTok{\textless{}{-}} \FunctionTok{open\_dataset}\NormalTok{(}\StringTok{"cvrs"}\NormalTok{)}
\end{Highlighting}
\end{Shaded}

where \texttt{cvrs} indicates the path to the top-level folder
containing the parquet files. We have organized the parquet collection
as a hierarchy of contests nested within states. Users can see that the
folder has the following structure:

\begin{Shaded}
\begin{Highlighting}[]
\NormalTok{├── state=ARIZONA}
\NormalTok{│   ├── county\_name=MARICOPA}
\NormalTok{│   │   └── part{-}0.parquet}
\NormalTok{│   ├── county\_name=PIMA}
\NormalTok{│   │   └── part{-}0.parquet}
\NormalTok{│   ├── county\_name=SANTA\%20CRUZ}
\NormalTok{│   │   └── part{-}0.parquet}
\NormalTok{│   └── county\_name=YUMA}
\NormalTok{│       └── part{-}0.parquet}
\NormalTok{...}
\NormalTok{── state=UTAH}
\NormalTok{│   └── county\_name=SAN\%20JUAN}
\NormalTok{│       └── part{-}0.parquet}
\NormalTok{└── state=WISCONSIN}
\NormalTok{    ├── county\_name=BROWN}
\NormalTok{    │   └── part{-}0.parquet}
\NormalTok{    ├── county\_name=KENOSHA}
\NormalTok{    │   └── part{-}0.parquet}
\NormalTok{    ├── county\_name=PIERCE}
\NormalTok{    │   └── part{-}0.parquet}
\NormalTok{    └── county\_name=WAUKESHA}
\NormalTok{        └── part{-}0.parquet}
\end{Highlighting}
\end{Shaded}

\texttt{open\_dataset()} links to all these subdirectories. Our data
release provides entire collection in a single zip file, but users may
still visualize the file hierarchy and track the collection's individual
files prior to download by using Dataverse's ``Preview'' feature.

\begin{verbatim}
<!-- Users just need to click the eye icon on the right hand side of the zip file to open the Preview and visualize the parquet hierarchy. -->
\end{verbatim}

Because parquet does not need to read all rows at once to count it, it
is much faster to provide summary statistics of subsets. Even though the
code below counts some 159 million rows, it performs the count in one
second on a personal laptop.

\begin{Shaded}
\begin{Highlighting}[]
\NormalTok{ds }\SpecialCharTok{|\textgreater{}} \FunctionTok{count}\NormalTok{(office) }\SpecialCharTok{|\textgreater{}} \FunctionTok{collect}\NormalTok{()}
\end{Highlighting}
\end{Shaded}

\begin{verbatim}
# A tibble: 6 x 2
  office              n
  <chr>           <int>
1 STATE SENATE 21568767
2 STATE HOUSE  37171773
3 US PRESIDENT 40690427
4 US SENATE    19061850
5 US HOUSE     40419744
6 GOVERNOR       562624
\end{verbatim}

In the arrow package, we use the \texttt{collect()} command to extract
the data. To perform the count, we use \texttt{count()} from
\texttt{dplyr}, which totals the number of occurrences of each unique
value in our \texttt{office} variable. We make use of R's pipe operator,
\texttt{\textbar{}\textgreater{}} to pass our data objects forward onto
subsequent operations we want to perform. All previous transformations
are \emph{lazily-loaded}, meaning that they are not executed until
needed. The arrow program combines the transformations internally in a
way that avoids duplicative operations.

\paragraph{Extracting Summaries}

We can typically use the combination of \texttt{state}, \texttt{office},
and \texttt{party} variable to identify candidates. The code below
counts the number of records for each candidate-party collection, sorted
from most frequent to least. With \texttt{filter()} from \texttt{dplyr},
we subset our count to presidential votes in the state of Wisconsin.

\begin{Shaded}
\begin{Highlighting}[]
\NormalTok{ds }\SpecialCharTok{|\textgreater{}} 
  \FunctionTok{filter}\NormalTok{(state }\SpecialCharTok{==} \StringTok{"WISCONSIN"}\NormalTok{, office }\SpecialCharTok{==} \StringTok{"US PRESIDENT"}\NormalTok{) }\SpecialCharTok{|\textgreater{}} 
  \FunctionTok{count}\NormalTok{(candidate, party, }\AttributeTok{sort =} \ConstantTok{TRUE}\NormalTok{) }\SpecialCharTok{|\textgreater{}} 
  \FunctionTok{collect}\NormalTok{()}
\end{Highlighting}
\end{Shaded}

\begin{verbatim}
# A tibble: 8 x 3
  candidate       party      n
  <chr>           <chr>  <int>
1 DONALD J TRUMP  REP   293283
2 JOSEPH R BIDEN  DEM   221356
3 JO JORGENSEN    LBT     6272
4 WRITEIN         W-I     1412
5 UNDERVOTE       OTH     1337
6 BRIAN T CARROLL OTH      874
7 DON BLANKENSHIP OTH      724
8 OVERVOTE        OTH      493
\end{verbatim}

For individual voters, use the \texttt{cvr\_id} variable. This is a
numeric variable that is defined within counties. The following code
extracts the vote from the voter marked with the
\texttt{cvr\_id\ ==\ 1}. By setting the \texttt{sort} argument to
\texttt{TRUE}, we ensure the output is sorted by most to least frequent
value. These numbers do not in any way indicate the time in which the
ballot was cast, or the personal identity of the voter. However, it does
show that this voter split their ticket, voting for Democrats in the
Presidential and Congressional race, while voting for one Republican
candidate in state senate.

\begin{Shaded}
\begin{Highlighting}[]
\NormalTok{ds }\SpecialCharTok{|\textgreater{}} 
  \FunctionTok{filter}\NormalTok{(state }\SpecialCharTok{==} \StringTok{"ARIZONA"}\NormalTok{, county\_name }\SpecialCharTok{==} \StringTok{"MARICOPA"}\NormalTok{) }\SpecialCharTok{|\textgreater{}} 
  \FunctionTok{filter}\NormalTok{(cvr\_id }\SpecialCharTok{==} \DecValTok{1}\NormalTok{) }\SpecialCharTok{|\textgreater{}} 
  \FunctionTok{select}\NormalTok{(county\_name, cvr\_id, office, district, candidate, party) }\SpecialCharTok{|\textgreater{}} 
  \FunctionTok{collect}\NormalTok{()}
\end{Highlighting}
\end{Shaded}

\begin{verbatim}
# A tibble: 6 x 6
  county_name cvr_id office       district candidate        party
  <chr>        <int> <chr>        <chr>    <chr>            <chr>
1 MARICOPA         1 US PRESIDENT FEDERAL  JOSEPH R BIDEN   DEM  
2 MARICOPA         1 US SENATE    ARIZONA  KELLY MARK       DEM  
3 MARICOPA         1 US HOUSE     008      MUSCATO MICHAEL  DEM  
4 MARICOPA         1 STATE SENATE 013      KERR SINE        REP  
5 MARICOPA         1 STATE HOUSE  013      DUNN TIMOTHY TIM REP  
6 MARICOPA         1 STATE HOUSE  013      SANDOVAL MARIANA DEM  
\end{verbatim}

However, a further investigation into this voter's state senate district
shows that this was uncontested by a Democratic candidate. Therefore,
the voter had no choice to vote a fully straight ticket. The code shows
that contests are defined by state, office, and district number.

\begin{Shaded}
\begin{Highlighting}[]
\NormalTok{ds }\SpecialCharTok{|\textgreater{}} 
  \FunctionTok{filter}\NormalTok{(state }\SpecialCharTok{==} \StringTok{"ARIZONA"}\NormalTok{, office }\SpecialCharTok{==} \StringTok{"STATE SENATE"}\NormalTok{, district }\SpecialCharTok{==} \StringTok{"013"}\NormalTok{) }\SpecialCharTok{|\textgreater{}} 
  \FunctionTok{count}\NormalTok{(candidate, party) }\SpecialCharTok{|\textgreater{}} 
  \FunctionTok{collect}\NormalTok{()}
\end{Highlighting}
\end{Shaded}

\begin{verbatim}
# A tibble: 7 x 3
  candidate       party     n
  <chr>           <chr> <int>
1 "KERR SINE"     REP   93388
2 "NOT QUALIFIED" OTH    1852
3 "BACKUS BRENT"  OTH     145
4 "UNDERVOTE"     OTH    8006
5 "WRITEIN"       W-I     531
6 "OVERVOTE"      OTH       3
7 ""              OTH     119
\end{verbatim}

\paragraph{Application: Biden and Trump's Party Loyalty}

As our main exercise, we ask whether partisans --- defined by their
votes for Congress and state legislature --- vote for their party's
presidential candidate. Trump was a polarizing candidate. Election
observers have wondered if Trump drew less support from Republican
voters compare to Biden's support among Democratic voters.

For this analysis, we look at the counties in five battleground states
which together decided the election: Wisconsin, Michigan, Georgia,
Arizona, and Nevada.

\begin{Shaded}
\begin{Highlighting}[]
\NormalTok{ds\_states }\OtherTok{\textless{}{-}}\NormalTok{ ds }\SpecialCharTok{|\textgreater{}} 
  \FunctionTok{filter}\NormalTok{(state }\SpecialCharTok{\%in\%} \FunctionTok{c}\NormalTok{(}\StringTok{"WISCONSIN"}\NormalTok{, }\StringTok{"MICHIGAN"}\NormalTok{, }\StringTok{"GEORGIA"}\NormalTok{, }\StringTok{"ARIZONA"}\NormalTok{, }\StringTok{"NEVADA"}\NormalTok{))}
\end{Highlighting}
\end{Shaded}

While aggregate election results report how many votes Biden and Trump
received, they do not reveal which of those votes came from Republicans
and Democratic voters. As the previous example showed, the cast vote
records reveal exactly how each individual voter voted on the sets of
offices available. We therefore classify voters into (non-Presidential)
partisans based on how they voted in all offices except President.

We first need to narrow down our data so that we only use voter-contest
pairs in contests contested by a Democrat and a Republican. In other
words, the voter needed to have a choice to vote either way.

\begin{Shaded}
\begin{Highlighting}[]
\NormalTok{ds\_contested }\OtherTok{\textless{}{-}}\NormalTok{ ds\_states }\SpecialCharTok{|\textgreater{}} 
  \FunctionTok{collect}\NormalTok{() }\SpecialCharTok{|\textgreater{}} 
  \CommentTok{\# Contested contests}
  \FunctionTok{filter}\NormalTok{(}\FunctionTok{any}\NormalTok{(party }\SpecialCharTok{==} \StringTok{"REP"}\NormalTok{) }\SpecialCharTok{\&} \FunctionTok{any}\NormalTok{(party }\SpecialCharTok{==} \StringTok{"DEM"}\NormalTok{), }
         \AttributeTok{.by =} \FunctionTok{c}\NormalTok{(state, office, district)) }\SpecialCharTok{|\textgreater{}} 
  \CommentTok{\# Ballots with Presidential vote}
  \FunctionTok{filter}\NormalTok{(}\FunctionTok{any}\NormalTok{(office }\SpecialCharTok{==} \StringTok{"US PRESIDENT"}\NormalTok{), }
         \AttributeTok{.by =} \FunctionTok{c}\NormalTok{(state, county\_name, cvr\_id))}
\end{Highlighting}
\end{Shaded}

The first \texttt{filter()} command in the above snippet limits to
contested rows. For each state x office x district combination, we
examine if there are any Republican candidates \emph{and} any Democrats.
Contests that do not meet this criteria are dropped. The second
\texttt{filter()} command limits to ballots with a Presidential choice.
This excludes fragmented ballots where the President and the rest of the
ballot is separated. Both commands are done after \texttt{collect()}
because the \texttt{arrow} package does not support group-specific
filter commands as of the current version.

\noindent We now construct a dataset where each row is a single voter.
We first create a table of Presidential votes:

\begin{Shaded}
\begin{Highlighting}[]
\DocumentationTok{\#\# Voters based on President}
\NormalTok{ds\_pres }\OtherTok{\textless{}{-}}\NormalTok{ ds\_contested }\SpecialCharTok{|\textgreater{}} 
  \FunctionTok{filter}\NormalTok{(office }\SpecialCharTok{==} \StringTok{"US PRESIDENT"}\NormalTok{) }\SpecialCharTok{|\textgreater{}} 
  \FunctionTok{select}\NormalTok{(}
\NormalTok{    state, county\_name, }
\NormalTok{    cvr\_id, candidate,}
    \AttributeTok{pres\_party =}\NormalTok{ party) }\SpecialCharTok{|\textgreater{}} 
  \FunctionTok{mutate}\NormalTok{(}\AttributeTok{pres =} \FunctionTok{case\_when}\NormalTok{(}
\NormalTok{    pres\_party }\SpecialCharTok{==} \StringTok{"REP"} \SpecialCharTok{\textasciitilde{}} \StringTok{"Trump"}\NormalTok{, }
\NormalTok{    pres\_party }\SpecialCharTok{==} \StringTok{"DEM"} \SpecialCharTok{\textasciitilde{}} \StringTok{"Biden"}\NormalTok{, }
\NormalTok{    pres\_party }\SpecialCharTok{==} \StringTok{"LBT"} \SpecialCharTok{\textasciitilde{}} \StringTok{"Libertarian"}\NormalTok{, }
\NormalTok{    candidate }\SpecialCharTok{==} \StringTok{"UNDERVOTE"} \SpecialCharTok{\textasciitilde{}} \StringTok{"Undervote"}\NormalTok{,}
    \AttributeTok{.default =} \StringTok{"Other"}\NormalTok{))}
\end{Highlighting}
\end{Shaded}

Separately, we construct a dataset that classifies the same voters based
on their non-Presidential vote choice. The variable
\texttt{nonpres\_party} is \texttt{Down-ballot\ DEM} if the voter only
votes for Democrats down-ballot (using the \texttt{all()} command) and
it is \texttt{Down-ballot\ REP} if the voter only votes for Republicans
down-ballot.

\begin{Shaded}
\begin{Highlighting}[]
\DocumentationTok{\#\# subset to all{-}Dem voters based on everything except President}
\NormalTok{ds\_D }\OtherTok{\textless{}{-}}\NormalTok{ ds\_contested }\SpecialCharTok{|\textgreater{}} 
  \FunctionTok{filter}\NormalTok{(office }\SpecialCharTok{!=} \StringTok{"US PRESIDENT"}\NormalTok{) }\SpecialCharTok{|\textgreater{}} 
  \FunctionTok{filter}\NormalTok{(}\FunctionTok{all}\NormalTok{(party }\SpecialCharTok{==} \StringTok{"DEM"}\NormalTok{), }\AttributeTok{.by =} \FunctionTok{c}\NormalTok{(state, county\_name, cvr\_id)) }\SpecialCharTok{|\textgreater{}} 
  \FunctionTok{distinct}\NormalTok{(state, county\_name, cvr\_id) }\SpecialCharTok{|\textgreater{}} 
  \FunctionTok{mutate}\NormalTok{(}\AttributeTok{nonpres\_party =} \StringTok{"Down{-}ballot DEM"}\NormalTok{)}

\DocumentationTok{\#\# same subset, but for all{-}Rep voters}
\NormalTok{ds\_R }\OtherTok{\textless{}{-}}\NormalTok{ ds\_contested }\SpecialCharTok{|\textgreater{}} 
  \FunctionTok{filter}\NormalTok{(office }\SpecialCharTok{!=} \StringTok{"US PRESIDENT"}\NormalTok{) }\SpecialCharTok{|\textgreater{}} 
  \FunctionTok{filter}\NormalTok{(}\FunctionTok{all}\NormalTok{(party }\SpecialCharTok{==} \StringTok{"REP"}\NormalTok{), }\AttributeTok{.by =} \FunctionTok{c}\NormalTok{(state, county\_name, cvr\_id)) }\SpecialCharTok{|\textgreater{}} 
  \FunctionTok{distinct}\NormalTok{(state, county\_name, cvr\_id) }\SpecialCharTok{|\textgreater{}} 
  \FunctionTok{mutate}\NormalTok{(}\AttributeTok{nonpres\_party =} \StringTok{"Down{-}ballot REP"}\NormalTok{)}
\end{Highlighting}
\end{Shaded}

Now we join the Presidential data with the non-Presidential
classifications. Because each row is a voter, we perform a one to one
match using \texttt{cvr\_id} to link back the two office choices. Voters
who were not classified into DEM or REP, are, by construction, voters
who voted for some Democratic down-ballot candidates and Republican
down-ballot candidates. We label these voters
\texttt{nonpres\_party\ =\ Mixed}.

\begin{Shaded}
\begin{Highlighting}[]
\NormalTok{ds\_analysis }\OtherTok{\textless{}{-}}\NormalTok{ ds\_pres }\SpecialCharTok{|\textgreater{}} 
  \FunctionTok{left\_join}\NormalTok{(}
    \FunctionTok{bind\_rows}\NormalTok{(ds\_D, ds\_R), }
    \AttributeTok{by =} \FunctionTok{c}\NormalTok{(}\StringTok{"state"}\NormalTok{, }\StringTok{"county\_name"}\NormalTok{, }\StringTok{"cvr\_id"}\NormalTok{), }\AttributeTok{relationship =} \StringTok{"one{-}to{-}one"}\NormalTok{) }\SpecialCharTok{|\textgreater{}} 
  \FunctionTok{mutate}\NormalTok{(}\AttributeTok{nonpres\_party =} \FunctionTok{replace\_na}\NormalTok{(nonpres\_party, }\StringTok{"Mixed"}\NormalTok{))}
\end{Highlighting}
\end{Shaded}

\noindent Finally, we construct a cross-tabulation of this dataset using
the base-R \texttt{xtabs()} function.

\begin{Shaded}
\begin{Highlighting}[]
\FunctionTok{xtabs}\NormalTok{(}\SpecialCharTok{\textasciitilde{}}\NormalTok{ nonpres\_party }\SpecialCharTok{+}\NormalTok{ pres, ds\_analysis) }\SpecialCharTok{|\textgreater{}} 
  \FunctionTok{addmargins}\NormalTok{()}
\end{Highlighting}
\end{Shaded}

\begin{verbatim}
                 pres
nonpres_party       Biden Libertarian   Other   Trump Undervote     Sum
  Down-ballot DEM 3009174       15308    6712   36090      2841 3070125
  Down-ballot REP   65281       26452    9031 3204959      6358 3312081
  Mixed            651944       61704   13819  494828     10481 1232776
  Sum             3726399      103464   29562 3735877     19680 7614982
\end{verbatim}

This table shows for example that among 0 solidly Democratic voters, 0
voted for Joe Biden.\\
We can show cell counts in terms of proportions of the entire row, with
the following operation:

\begin{Shaded}
\begin{Highlighting}[]
\NormalTok{xtprop }\OtherTok{\textless{}{-}} \FunctionTok{xtabs}\NormalTok{(}\SpecialCharTok{\textasciitilde{}}\NormalTok{ nonpres\_party }\SpecialCharTok{+}\NormalTok{ pres, ds\_analysis) }\SpecialCharTok{|\textgreater{}} 
  \FunctionTok{prop.table}\NormalTok{(}\AttributeTok{margin =} \DecValTok{1}\NormalTok{) }\SpecialCharTok{|\textgreater{}} 
  \FunctionTok{round}\NormalTok{(}\DecValTok{3}\NormalTok{) }

\DocumentationTok{\#\# add margins}
\NormalTok{N }\OtherTok{\textless{}{-}} \FunctionTok{xtabs}\NormalTok{(}\SpecialCharTok{\textasciitilde{}}\NormalTok{ nonpres\_party, ds\_analysis)}

\DocumentationTok{\#\# reorder columns and append totals}
\NormalTok{xtprop[, }\FunctionTok{c}\NormalTok{(}\StringTok{"Biden"}\NormalTok{, }\StringTok{"Trump"}\NormalTok{, }\StringTok{"Libertarian"}\NormalTok{, }\StringTok{"Other"}\NormalTok{, }\StringTok{"Undervote"}\NormalTok{)] }\SpecialCharTok{|\textgreater{}} 
  \FunctionTok{cbind}\NormalTok{(}\FunctionTok{format}\NormalTok{(N, }\AttributeTok{big.mark =} \StringTok{","}\NormalTok{)) }\SpecialCharTok{|\textgreater{}} 
\NormalTok{  kableExtra}\SpecialCharTok{::}\FunctionTok{kbl}\NormalTok{(}\AttributeTok{format =} \StringTok{"latex"}\NormalTok{, }\AttributeTok{booktabs =} \ConstantTok{TRUE}\NormalTok{)}
\end{Highlighting}
\end{Shaded}

\begin{tabular}[t]{lllllll}
\toprule
  & Biden & Trump & Libertarian & Other & Undervote & \\
\midrule
Down-ballot DEM & 0.98 & 0.012 & 0.005 & 0.002 & 0.001 & 3,070,125\\
Down-ballot REP & 0.02 & 0.968 & 0.008 & 0.003 & 0.002 & 3,312,081\\
Mixed & 0.529 & 0.401 & 0.05 & 0.011 & 0.009 & 1,232,776\\
\bottomrule
\end{tabular}

This formatted table shows more clearly that the ``ticket splitting''
rate among solid partisans was on the order of 1 percent in this sample.
Such small samples are almost impossible to detect in a survey. In
contrast, 97 percent of solid Republicans stuck with their party's
nominee, Trump, and 98 percent of solid Democrats stuck with Biden. This
suggests that Trump's party loyalty was only a 1 percentage point
smaller than Biden's. However, a starker difference arises in the mixed
group (those who vote for some Republicans and some Democrats
down-ballot). Biden won this group of weak partisans by close to 10
points.

More can be done to examine if these results vary by state, county, or
precinct. Future versions of this dataset can also include ballot
measures and local candidates that give more context of these patterns.



\end{document}
